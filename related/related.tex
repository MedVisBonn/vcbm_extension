\section{Related Work}\label{related}
The diffusion MRI tractography pipeline suffers from several sources of
uncertainty \cite{Schultz:SciVisBook2014, Schultz:NBM2018, Gillmann:STAR2021}.
They can broadly categorised into the three categories measurement uncertainty,
model uncertainty and parameter uncertainty. 

The impact of measurement
uncertainty on the dMRI pipeline has been studied great detail. Within the last
decade there have been several methods proposed to simulate uncertainty within
the dMRI context, like Bayesian modeling \cite{BEHRENS2007144} or bootstrapping
\cite{Chung:2006}. Instead of single streamlines we then recover distributions,
which can be visualized using hyperstreamlines \cite{Jeurissen:2012, Wiens:2014}
or illustrative confidence intervals \cite{Brecheisen:2013}. Especially
bootstrapping tends to be very time consuming.

We refer to model uncertainty as the uncertainty, which arises from the choice
of several mathematical models to extract directions from the dMRI data
\cite{Schultz:SciVisBook2014}. It exists a wide range of such models to estimate
fiber direction \cite{Panagiotaki:2012}, and they might lead to different
results. There is no general preferabale model, since the suitability depends  on
the anatomical location \cite{Bretthorst:2004,Freidlin:2007}. Our previous work
can be seen as special case of model uncertainty, since we use a different model
for each fiber direction count \cite{Gruen:2021}. 

Another uncertainty arises from the parameter choice within the pipeline. In the
tracking algorithm we have to adjust several parameters, which have a great
impact on the final tractography, like a angle below which the streamline
branches, a maximum branching number, a minimum distance between two branches,
and overall parameters like stopping criterias. The last topic has been addressed by a visual
tool to explore the impact of such parameters in a systematical way
\cite{Brecheisen:2009}.


