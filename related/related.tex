\section{Related Work}\label{related}
The diffusion MRI tractography pipeline suffers from several sources of
uncertainty \cite{Schultz:SciVisBook2014, Schultz:NBM2018, Gillmann:STAR2021}.
They can broadly categorised into measurement uncertainty,
model uncertainty, parameter uncertainty, and partial voluming. 

The impact of measurement
uncertainty on the dMRI pipeline has been studied in greatest detail. Within the last
decade there have been several methods proposed to simulate uncertainty within
the dMRI context, like Bayesian modeling \cite{BEHRENS2007144} or bootstrapping
\cite{Chung:2006}. Instead of single streamlines we recover distributions,
which can be visualized using hyperstreamlines \cite{Jeurissen:2012, Wiens:2014}
or illustrative confidence intervals \cite{Brecheisen:2013}. Especially
bootstrapping tends to be very time consuming.

We refer to model uncertainty as the uncertainty, which arises from the choice
of several mathematical models to extract directions from the dMRI data
\cite{Schultz:SciVisBook2014}. There exists a wide range of models to estimate
fiber direction \cite{Panagiotaki:2012}, and they might lead to different
results. There is no general preferabale model, since the suitability depends  on
the anatomical location \cite{Bretthorst:2004,Freidlin:2007}. Our previous work
can be seen as a special case of model uncertainty, since we use a different model
for each fiber direction count \cite{Gruen:2021}. 

Another uncertainty arises from the parameter choice within the pipeline. In the
tracking algorithm we have to adjust several parameters, which have a great
impact on the final tractography, like a threshold angle below which the streamline
branches, a maximum branching number, a minimum distance between two branches,
and overall parameters like stopping criterias. Brecheisen et al. have proposed
a visual
tool to explore the impact of the parameters in a systematical way
\cite{Brecheisen:2009}.

Lately the focus of uncertainty in dMRI has been set to the partial volume
effect. While the resolution of a modern dMRI is restricted to 1.25 mm in each
direction, the size of an axon is around 0.1-20 $\mu m$. Therefore, in each voxel
are many axons present and we can only measure an average of all these axons.
This leads to two problems. Firstly, a tract can spread within a voxel. This can
be partially solved by modern crossing fiber approaches and
tractography accounting for it using branching or a probabilistic direction
selection.
Secondly, two bundles can intersect
within a voxel and are then indistinguishable for local tracking approaches
\cite{Schilling:2022}. Within this work we use regions of interest, which are
based on our prior knowledge to reduce this source of uncertainty. A more
advanced approach to reduce this source of error is to use machine learning for
tract segmentation as it was done in Tracseg \cite{WASSERTHAL2018239}.


