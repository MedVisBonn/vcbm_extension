\section{Introduction}
Diffusion Magnetic Resonance Imaging (dMRI) \cite{LeBihan:1986} is a
non-invasive vital imaging method for the human brain. It particularly has the
unique ability to get insights into major white matter tracts within the human
brain and can
reconstruct large parts of the white matter tracts, using tractography algorithms. dMRI is based on
inferring the fiber tracts using the Brownian motion of water molecules. Since
the fiber tracts impede the water movement it will move along the fiber tracts
rather than orthogonal to it. 
This unique ability of fiber tracking makes dMRI an important feature for large
scientific studies \cite{Sotiropoulos:2013, Tobisch:2018Frontiers} as well as surgery planning \cite{Yang:2021}.

The process of obtaining tractograms is not trivial at all and contains many
pitfalls. The most popular and widely used approaches recover the local
orientation of fiber tracts from dMRI measurements. This is an ill
conditioned inverse problem \cite{TOURNIER20071459}. Earlier approaches, like
diffusion tensor imaging \cite{BASSER1994247}
were just able to recover a single direction per voxel, which is not sufficient
to recover more complex geometries like fibre crossing, kissing and bending.
Newer approaches rely on high angular resolution imaging (HARDI). Within the
last decade there have been many methods invented to compute multiple local orientations
from raw diffusion data, including the
ball-and-stick model \cite{BEHRENS2007144}, and the low-rank approximation of high order fODF tensors
\cite{lowrank, Ankele:CARS2017}.

Within our last work we have evaluated the influence of model uncertainty in
the low-rank approximation in a systematical way. To be more precise, the fiber
count has to be set a priori to apply the low-rank approximation. This is a
crucial step, since setting the number to low will miss relevant directions and
setting the number to high will introduce wrong directions which makes the
tracking process even more challenging. Further the remaining directions are
more error-prone, if the number is not set correctly. 
Therefore, we estimated the number of fibers with the help of a Bayesian model and
created a selection model from these estimations. To improve the precision of
the local direction fields
further, we proposed a novel averaging model, which fuses all the different
low-rank approximations into a new model and therefore, reduces this source of
uncertainty even further.
Using a probabilistic fiber tracking model, it has been shown that fiber
tracking within the newly proposed models is more robust and results in
better reconstructions compared to fiber tracking within a state of the art
 $8$th constrained spherical deconvolution model.

With the increasing complexity of scanner protocols and the resulting more
complex models also the sensitivity to measurement noise increases. Within this work
we will systematically investigate the influence of noise to the generated
models. We use a wild bootstrapping approach, which has been
successfully applied to dMRI in many cases \cite{Jones:2008}. Instead of
calculating a fiber distribution to visualize uncertainty, we proceed in
another way and evaluate the impact of noise to the average, selection, and
low-rank model.

We are using the newly derived quantification of measurement uncertainty to the
direction fields to generate a novel consensus bootstrap
model, which fuses all the bootstrap information into a single model. This new
approach is then compared to the previously derived models as well as the
rank-$3$ model. We conclude that the
impact on crossing fiber tractography depends highly on the model. While the
improvement in the selection model is huge, the improvement of the
average model is not large. 

In Section \ref{related} we will introduce some related work which will help the
reader to put our work into the right context. This is followed in Section \ref{background} by an
explanation of the model on which our approach is based.
