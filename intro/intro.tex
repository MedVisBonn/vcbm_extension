\section{Introduction}
Diffusion Magnetic Resonance Imaging (dMRI) \cite{LeBihan:1986} is a
non-invasive imaging method for the human brain. It allows for
unique insights into the geometry and microstructure of major white matter tracts by measuring the Brownian motion of water molecules. Since
the fiber tracts impede their movement, molecules move along the fiber tracts
more freely than orthogonal to it. This has permitted the reconstruction of many white matter tracts using tractography algorithms, and has made dMRI an important tool for large
scientific studies \cite{Sotiropoulos:2013, Tobisch:2018Frontiers} as well as surgery planning \cite{Yang:2021}.

The most popular and widely used tractography algorithms recover the local
orientation of fiber tracts from dMRI measurements. Earlier approaches like
Diffusion Tensor Imaging \cite{BASSER1994247}
were just able to recover a single direction per voxel, which is not sufficient
to recover more complex geometries like fiber crossing, kissing, and bending.
Newer approaches rely on high angular resolution imaging (HARDI), and typically lead to ill-conditioned inverse problems. Popular mathematical models for the estimation of multiple local orientations include the
ball-and-stick model \cite{BEHRENS2007144}, spherical deconvolution \cite{TOURNIER20071459}, and the low-rank approximation of higher order fODF tensors
\cite{lowrank}, which can be seen as combining aspects of the first two \cite{Schultz:MICCAI10}.

Tractography is affected by many sources of uncertainty \cite{Schultz:SciVisBook2014}. One of them is the model uncertainty that arises when having to make an \emph{a priori} choice of the number of fibers in a given voxel. In models such as ball-and-sticks or low-rank tensor approximation, this is a
crucial step, since setting the number too low will miss relevant directions, while setting it too high will introduce spurious directions and increase variance in the remaining directions.

We recently introduced the first framework for the quantification of this type of uncertainty, and a method for its reduction, which we refer to as \emph{model averaging} \cite{Gruen:2021}. Our current work reports a refined implementation of that idea, and makes the following additional contributions:
\begin{enumerate}
\item We study how the above-described model uncertainty interacts with data uncertainty, which arises due to measurement noise and is commonly estimated via bootstrapping \cite{Chung:2006}.
\item We design a new approach for the joint reduction of data and model uncertainty based on a bootstrap consensus strategy, and compare it to our previous model averaging technique.
\item We evaluate a combination of both ideas and compare it to results from a novel baseline, which simply estimates the maximum number of fibers everywhere.
\end{enumerate}

% Maybe these details can come later?
%Therefore, we estimated the number of fibers with the help of a Bayesian model and
%created a selection model from these estimations. To improve the precision of
%the local direction fields
%further, we proposed a novel averaging model, which fuses all the different
%low-rank approximations into a new model and therefore, reduces this source of
%uncertainty even further.
%Using a probabilistic fiber tracking model, it has been shown that fiber
%tracking within the newly proposed models is more robust and results in
%better reconstructions compared to fiber tracking within a state of the art
% $8$th constrained spherical deconvolution model.

% With the increasing complexity of scanner protocols and the resulting more
% complex models also the sensitivity to measurement noise increases. Within this work
% we will systematically investigate the influence of noise to the generated
% models. We use a wild bootstrapping approach, which has been
% successfully applied to dMRI in many cases \cite{Jones:2008}. Instead of
% calculating a fiber distribution to visualize uncertainty, we proceed in
% another way and evaluate the impact of noise to the average, selection, and
% low-rank model.

% We are using the newly derived quantification of measurement uncertainty to the
% direction fields to generate a novel consensus bootstrap
% model, which fuses all the bootstrap information into a single model. This new
% approach is then compared to the previously derived models as well as the
% rank-$3$ model. We conclude that the
% impact on crossing fiber tractography depends highly on the model. While the
% improvement in the selection model is huge, the improvement of the
% average model is not large. 

The remainder of our paper is organized as follows: We first provide the required background by discussing related work in general (Sections~\ref{sec:related}), and briefly summarizing the crossing fiber model on which our work is based (Section~\ref{sec:background}). We then introduce our two main approaches, model averaging and bootstrap consensus, in Sections~\ref{sec:Models} and~\ref{sec:bootstrap-consensus}, respectively, and describe an algorithm for dMRI tractography that uses them in  Section~\ref{sec:tracking}. We report and discuss our experimental results in Section~\ref{sec:results}, before Section~\ref{sec:conclusion} summarizes our main findings and concludes our work.

%%% Local Variables:
%%% mode: latex
%%% TeX-master: "../main"
%%% End:
