\section{Conclusion}
\label{sec:conclusion}
Within this work we have evaluated the impact of measurement noise to the
average, selection and low-rank model. It has been shown, that the selection as
well as the average model are more resistant against noise than the low-rank
model. 
We have used the introduced uncertainty
to build a consensus model, which shows for the selection and average model
significantly better results compared to their base models. 
The highest increase in average Dice score is visible for the consensus
selection model compared to the base selection model. Compared to the average
model, the increasement is not that large, but significant. Together with the
huge computational effort, which is needed to create the consensus model
compared to the easieness of the base average model, we conclude that the
average model is able to increase the multi vector fields greatly at low cost
and is therefore, in almost all cases preferable. 

\section*{Acknowledgments}

We thank Gemma van der Voort (University of Bonn) for her initial proof-of-concept implementation of tractography with model averaging, and for providing feedback on our manuscript. This work was funded by the Deutsche Forschungsgemeinschaft (DFG, German Research Foundation) -- 422414649.
Data were provided by the Human Connectome Project, WU-Minn Consortium (Principal Investigators: David Van Essen and Kamil Ugurbil; 1U54MH091657) funded by the 16 NIH Institutes and Centers that support the NIH Blueprint
for Neuroscience Research; and by the McDonnell Center for Systems Neuroscience at Washington University.