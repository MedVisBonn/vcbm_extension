\section{Background}
\begin{itemize}
	\item Introduce spherical deconvolution since it is the basis for
		bootstrapping. 

\end{itemize}
Spherical deconvolution is based on solving the linear least square problem 
\begin{align}
	\argmin_{\mathcal{T}} \| M \mathcal{T} - S \|^2,
	\label{eq:sd-min}
\end{align}
where 
$M$ denotes the convolution matrix, $S$ the signal and $\mathcal{T}$ the fODF.
Earlier approaches expressed the fODF as spherical function and imposed a
non-negativity constraint to Eq. (\ref{eq:sd-min}). The constrained is justified by
the fact, that the fODF should represent the fiber fraction in any given
direction, which is clearly positive. 
Newer approaches express the fODF as symmetric fourth order tensor and apply a
H-psd constrained, which implies non-negativity but not vice versa. It has been
demonstrated, that latter expression increases the angular resolution. 

From the symmetric fourth order fODF $\mathcal{T}f$ peaks are extracted via a rank-$r$
approximation 
\begin{align}
	\mathcal{T}^{\left( r \right)} = \sum_{i=1}^r \lambda_i \mathbf{v}_i
	\otimes \mathbf{v}_i \otimes \mathbf{v}_i \otimes \mathbf{v}_i, 
	\label{eq:low-rank}
\end{align}
where the scalar $\lambda_i$ represents the volume fraction of the $i$th fiber,
the unit vector $\mathbf{v}_i \in \mathbb{R}^3$ its direction and $\otimes$ the
outer product. The approximation is done by calculating 
\[ \argmin_{\lambda_1, \dots , \lambda_r , \mathbf{v}_1, \dots , \mathbf{v}_r}
\| \mathcal{T} - \mathcal{T}^{\left( r \right)} \|_F, \]
where $\| \cdot \|_F$ denotes the Frobenius norm. In case of fiber crossings the
used methods have a highly reduced angular error, as it was demonstrated in XY,
compared to peak extraction, which is applied to spherical function fODFs. 


