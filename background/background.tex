\section{Background}\label{background}
Earliest approaches used a diffusion tensor model
\cite{LeBihan:1986,Mori:1999} to estimate a single dominant fiber orientation
per voxel. From an anatomical point of view it is known that large parts of the
white matter voxel contain more than one fiber bundle \cite{Jeurissen:2012}, and it has also be shown
that accounting for these more complex fiber geometries improve the dMRI
tractography greatly \cite{Neher:2015}

Therefore, we use constrained spherical deconvolution, which is a widely used
mathematical model that accounts for multiple fibers orientations.
Spherical deconvolution is based on solving the linear least square problem 
\begin{align}
	\argmin_{\mathcal{T}} \| M \mathcal{T} - S \|^2,
	\label{eq:sd-min}
\end{align}
where 
$M$ denotes the convolution matrix, $S$ the signal and $\mathcal{T}$ the fiber
orientation distribution function (fODF).
Earlier approaches expressed the fODF as spherical function and imposed a
non-negativity constraint to Eq. (\ref{eq:sd-min}) \cite{TOURNIER20071459}. The constrained is justified by
the fact, that the fODF should represent the fiber fraction in any given
direction, which is clearly positive. Crossing fiber tractography can now be
performed by following in the direction of local fODF maxima.

Newer approaches express the fODF as symmetric fourth order tensor and apply a
H-psd constrained, which implies non-negativity but not vice versa. It has been
demonstrated, that latter expression increases the angular resolution.
Therefore, all following results use the fODF tensor representation
\cite{Ankele:CARS2017}. 

From the symmetric fourth order fODF $\mathcal{T}$ $r$ peaks are extracted via a rank-$r$
approximation 
\begin{align}
	\mathcal{T}^{\left( r \right)} = \sum_{i=1}^r \lambda_i \mathbf{v}_i
	\otimes \mathbf{v}_i \otimes \mathbf{v}_i \otimes \mathbf{v}_i, 
	\label{eq:low-rank}
\end{align}
where the scalar $\lambda_i$ represents the volume fraction of the $i$th fiber,
the unit vector $\mathbf{v}_i \in \mathbb{R}^3$ its direction and $\otimes$ the
outer product. The approximation is done by calculating 
\[ \argmin_{\lambda_1, \dots , \lambda_r , \mathbf{v}_1, \dots , \mathbf{v}_r}
\| \mathcal{T} - \mathcal{T}^{\left( r \right)} \|_F, \]
where $\| \cdot \|_F$ denotes the Frobenius norm. In case of fiber crossings the
used methods have a highly reduced angular error, as it was demonstrated by
Schultz et al. \cite{lowrank},compared to peak extraction, which is applied to spherical function fODFs.

Neither CSD nor the low-rank approximation is resistant against
measurement noise. The resistance against noise has not been evaluated for the
CSD with H-psd and low-rank $r$ approximation. However, from a theoretical
standpoint the low-rank approximation should be very sensitive to noise. While
directions with a high volume fraction are consistent, small directions should
be very sensitive.

\begin{itemize}
	\item Hier noch beschreiben, dass CSD schon measurement noise reduziert. 
	\item Vielleicht das ganze noch umsortieren. Gerade etwas durcheinander. 
\end{itemize}
