\section{Averaging and Selection Model}\label{sec:Models}

Within our work we have proposed a probabilistic framework
to quantify the posterior probability with which different potential fiber values
are suitable given a local fODF.
Using these information is especially for the low-rank approximation important,
since a wrong assumption on the number of fibers in a local voxel introduces new
fiber directions or misses important fibre directions. Which then leads to false
positive or false negatives. Further, the fibers are not independent of each
other, i.e. missing the correct number will also introduce an error to the
fibers.

\subsection{Bayesian Model Comparison}
We use a Bayesian model comparison approach, i.e. we are interested in the
posterior probability $p \left( \mathcal{H}_r \mid \mathcal{T} \right)$, where
$\mathcal{H}_r$ denotes the hypothesis that rank $r$ is the optimal rank to
extract $r$ fibers from a given fODF $\mathcal{T}$. Using Bayes' theorem of
conditional probability, the posterior probability can be rewritten as
\begin{align}
	p \left( \mathcal{H}_r \mid \mathcal{T} \right) \propto p \left(
		\mathcal{T} \mid \mathcal{H}_r 
	\right) p \left(  \mathcal{H}_r \right), 
	\label{eq:Bayes}
\end{align}
where $p \left(  \mathcal{H}_r \right)$ is our prior belief that rank $r$ is
suitable, without considering the fODF. While in literature the agreement
towards multi-fiber configurations is strong, the distribution over the brain
differs from source to source  \cite{BEHRENS2007144,Jeurissen:2012, Schultz:MICCAI12}. We have evaluated several  prior values from literature,
but the effect on the posteriors were minor. Therefore, we decided to use a
non-informative prior that assigns equal prior probability to the values of $r
\in \left\{ 1,2,3 \right\}$. The case $r=0$ can be excluded since we limit
tracking to a white matter mask. 

The remaining term $p \left( \mathcal{T} \mid \mathcal{H}_r \right)$, is the
probability of the fODF $\mathcal{T}$ given a rank $r$ and is in the context of
Bayesian model comparison referred as model evidence. It is derived from $p
\left( \mathcal{T} \mid \mathcal{H}_r , \Theta_r \right)$, the posterior
probability of $T$ given an $r$-fiber model with a specific parameter vector
$\Theta_r$. For the low-rank approximation $\mathcal{T}^{\left( r \right)}$ the
parameter vector is $\Theta_r \coloneqq \left( \lambda_1 , \mathbf{v}_1 , \dots
, \lambda_r , \mathbf{v}_r \right)$. 

The overall model evidence does not depend on any particular parameter values
and can therefore marginalized out 
\begin{align}
	p \left( \mathcal{T} \mid \mathcal{H}_r \right) = \int p \left(
		\mathcal{T} \mid \mathcal{H}_r , \Theta_r 
	\right) p \left( \Theta_r \mid \mathcal{H}_r  \right) d \Theta_r. 
	\label{eq:model-evidence}
\end{align}

Since the calculation of Eq. (\ref{eq:model-evidence}), would require to solve a
high dimensional integral, we use a approximation via the Bayesian Information
Criterion (BIC) 
\[ \text{BIC} = k \ln \left( n \right) - 2 \ln \left( p \left( \mathcal{T} \mid
\mathcal{H}_r, \hat{\Theta}_r \right) \right), \]
where $p \left(  \mathcal{T} \mid \mathcal{H}_r , \hat{\Theta}_r \right)$
corresponds to the likelihood of the rank-$r$ with parameters $\hat{\Theta}_r$
that best fit the fODF $\mathcal{T}$, $k$ is the number of parameters in
$\Theta_r$, and $n$ denotes the number of data points to which the model was
fitted \cite{Schwarz1978}. Under certain conditions the BIC is related to the
model evidence by \cite{Konishi2008}
\begin{align}
	p \left( \mathcal{T} \mid \mathcal{H}_r \right) \approx \exp \left(  -
		\frac{\text{BIC}}{2}
\right).
	\label{eq:BIC-model}
\end{align}
This allows us to compute the model evidence in a simple and efficent way and
will be used in the following work. 
\subsection{From Model Likelihood to Model Uncertainty}
To apply the Bayesian framework, we still need to provide an equation for $p
\left( \mathcal{T} \mid \mathcal{H}_r , \hat{\Theta}_r \right)$. Therefore, we
use the relative magintude of the corresponding low-rank approximation residual 
\begin{align}
	\| \tilde{\mathcal{R}}^{\left( r \right)} \| = \frac{ \| \mathcal{T} -
	\mathcal{T}^{\left( r \right)} \| }{ \| \mathcal{T} \|} \in \left[ 0,1
	\right].
	\label{eq:residual}
\end{align}
A zero residual would indicate a crossing fiber model that explains the observed
fODF perfectly, which is unrealistic in real world applications due to
measurement noise and other effects that our crossing fiber model does not
account for. Since modelling all these effects would be extremely challenging,
we use the Kumaraswamy Probabilistic Density Function (PDF)
\cite{Kumaraswamy1980}
\begin{align}
	f \left( x, a, b \right) \coloneqq ab x^{a-1} \left( 1- x^a
	\right)^{b-1} \text{ for } x \in \left( 0,1 \right) \text{ and } a,b >
	0.
	\label{eq:Kumaraswamy}
\end{align}

The Kumaraswamy PDF because it has two main benefits. Firstly, it
provides a considerable flexibility through its parameters $a$ and $b$ on the
interval. Secondly, it is computational more efficient than related
$\beta$-distributions. 

For our experiments we have set $p \left(  \mathcal{T} \mid \mathcal{H}_r,
\hat{\Theta}_r \right) = f \left(  \| \hat{\mathcal{R}}^{\left( r \right)} \| ; 1,20
\right)$. This leads to a monotonically decreasing probability as the relative
residual increases, and as it is visible in Fig CSDSF it supports our prior
assumptions about single fiber regions well.

\subsection{Computing Local Tracking Directions}
In previous work the strategy has been to determine an optimal rank $r \in
\left\{ 1,\dots , 3 \right\}$ in each integration step and use the set of
directions $\mathbf{v}_i$ for tracking \cite{Anekele:CARS2017}. Using this
approach overlooks the uncertainty which arises by the selection of a single
model, if several models have non-negligible probabilities. We use this approach
as direct comparison and use the model with the highest probability $p \left(
 \mathcal{H}_r \mid \mathcal{T} \right)$ according to the framework from the
 previous section.
 
As a new contribution, we propose to fuse the information of all different
models into a new averaging model. I.e. we create a weighted sum of
corresponding parameters $\mathbf{v}_i^{\left( r \right)}$ and
$\lambda_i^{\left( r \right)}$ from the different
$r$-fiber models with weights given by the posterior probabilities of the
models.
