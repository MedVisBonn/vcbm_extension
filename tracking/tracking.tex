\section{Crossing Fiber Tractography with Reduced Measurement Uncertainty}

\subsection{Probabilistic Streamline-Based Tractography}

Within this work a probabilistic streamline-based approach is used. For a given
seed point, the streamline grows iteratively in both directions using Euler
integration. Since neither the seed point nor any other point of the streamline
lies on a grid point, we interpolate the current position trilinearly. Therefore,
we have to solve a matching problem first, since we have to decide which
directions belong together to apply the trilinear interpolation. Since we assume
smoothness between the voxels we use the three directions, from the last
interpolation step as group means $\bar{\mathbf{v}}_i$, $i \in \left\{ 1\dots 3 
\right\}$. Now we fit each vertex to the group means such
that 
\begin{align}
	\text{Sym} \left( r  \right) & \rightarrow \mathbb{R}_+ \\ 
	Z & \mapsto \sum \| \sgn \left( \langle \mathbf{v}_{Z \left( i \right)}
	, \bar{\mathbf{v}}_i \rangle  \right) \mathbf{v}_{Z \left( i \right)} -
\bar{\mathbf{v}}_i
	\|
\end{align}
is minimized. We only update the group means if a group mean is not defined.
Then we solve the above minimization problem and assign the direction as new
group mean. Afterwards we rerun the minimization for all vertices. This is done,
to reduce the time consumption drastically. Further the local optimal
configuration is cached. Using this, we only have to calculate the assignments
once for each voxel which we are tracking trough. 
To initialize the trilinear interpolation we take the directions of the nearest voxel as group means.
Given the interpolated directions $\mathbf{v}_i$ at the current point, we
reorient them to have a non negative inner product with the current tracking
direction $\mathbf{w}$. We select one of the $r$ possible directions by the
following probabilistic model.

We assign each unit direction $\mathbf{v}_i$ with the volume fraction
$\lambda_i$ for $i \in \left\{ 1\dots r \right\}$ following the probability
scheme 
\[
	p \left( \mathbf{v}_i \right) \coloneqq \frac{ \mathbb{1}_{\lbrace\theta_i <
		\frac{1}{3} \pi \rbrace} \lambda_i \cos \left( \left( \frac{3}{\sqrt{2
\pi}} \theta_i \right)^2 \right)}{\sum_j \mathbb{1}_{\lbrace\theta_j <
		\frac{1}{3} \pi \rbrace} \lambda_j \cos \left( \left( \frac{3}{\sqrt{2
\pi}} \theta_j \right)^2 \right) }, 
\]
where $\theta_i$ denotes the angle between the possible direction $\mathbf{v}_i$
and the current direction $\mathbf{w}$. 

The proposed probability function assigns angles below 30 degrees almost the
same probability, which coincides with the limited angular resolution of
spherical deconvolution \cite{TOURNIER20071459}. Further the maximum angle is
restricted to 60 degrees, which coincides with our anatomical knowledge.

This iterative algorithm proceeds until, we either reach a region with fractional
anisotropy below $0.3$ or the summed angle over the last $30$ mm is greater than
$130$ degrees. This prevent streamlines to go back and forth. 

\subsection{Postprocessing}
While diffusion MRI is in general able to reconstruct large parts of the white
matter tracts, it is also well known that it is suspiscious to reconstruct false
positives, which have to be removed according to anatomic knowledge
\cite{Wakana:2007, MaierHein:2017}. Further the probabilistic tracking approach
generates outliers with low density, which have to be removed to have a proper
output. 

To avoid false positives, we use inclusion and exclusion regions. All regions
are set carefully for a reference patient according to [hier die quelle].
The remaining patients are linearly registered to the reference patient and the
linear transformation is used to map the regions to the other patients. 
If a Streamline intersect with a exclusion region the whole streamline is
discarded and if a streamline does not intersect with all inclusion regions it
is discarded as well. 

Further we create a density map for each streamline
bundle. Therefore, we count the number of streamlines intersecting each voxel.
All streamlines are cut of at the first intersection with a low density
area starting from the seed. The low density threshold is also defined for the
reference patient and then mapped to all other patients according to the seed
ratio.
