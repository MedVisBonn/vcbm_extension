\section{Results}
\subsection{Data}
The proposed novel bootstrap consensus model is applied to the selection,
average and rank $3$ model. The CSD model got replaced by the rank $3$ model
because it was knocked off compared to the other models by Dice score. Further,
the CSD used in the model is computational heavy, i.e. computation takes 4 times as long
as the $4$th order CSD which is used within this paper. This especially matters
because of the time intense bootstrap creation. 

To evaluate the proposed models, data from the Human
Connectome Project (HCP) is used \cite{HCP}. The diffusion MR images have a resolution of $1.25$ mm
isotropic with $145 \times 174 \times 145$ voxels. 

As reference data we use the high quality data, which was published within the
scope of the Tractseg paper \cite{WASSERTHAL2018239}. This reference data has been
created by manually refining the segmented full brain fiber tractography. For
more details we refer to the original literature.

All tests were performed on 12 randomly chosen subjects for which such reference tractographies
exist. For each tract we created a seeding plane by intersecting the reference
fiber bundle with a plane and initialize the tracking process with the direction
of the fiber bundle at the seedpoint. This should mimic a directional region of
interest as it might be defined by an expert on brain anatomy
\cite{Graumann2016}. We then apply the tracking process until we have as many
streamlines as the reference tractography. This should guaranty that the
comparison between the different models is fair . Therefore, we compare the approach by
Dice score, which is defined as
\[ 
	DICE = \frac{2 |RD \cap TR |}{|TR| + |RD|} ,
\]
where $RD$ denotes the reference data and $TR$ the tracking results
\cite{SCHILLING2019194}. 

\subsection{Qualitative Comparison}
\begin{figure}[t]
	\centering
	\begin{subfigure}[b]{0.45\linewidth}
		\includegraphics[width=\linewidth]{cst-rank-c}
		\caption{Low rank 3 model}
	\end{subfigure}
	\begin{subfigure}[b]{0.45\linewidth}
		\includegraphics[width=\linewidth]{cst-rank-bootstrap-c}
		\caption{Consensus low rank 3 model}
\end{subfigure} \\
	\begin{subfigure}[b]{0.45\linewidth}
		\includegraphics[width=\linewidth]{cst-avg-c}
		\caption{Average model}
	\end{subfigure}
	\begin{subfigure}[b]{0.45\linewidth}
		\includegraphics[width=\linewidth]{cst-avg-bootstrap-c}
		\caption{Consensus Average model}
\end{subfigure} \\
	\begin{subfigure}[b]{0.45\linewidth}
		\includegraphics[width=\linewidth]{cst-sel-c}
		\caption{Selection model}
	\end{subfigure}
	\begin{subfigure}[b]{0.45\linewidth}
		\includegraphics[width=\linewidth]{cst-sel-bootstrap-c}
		\caption{Consensus Selection Model}
\end{subfigure} \\
\caption{Reconstruction of the right Corticospinal Tract.}
	\label{fig:CST}
\end{figure}
As a first experiment we tracts the right Cortiscospinal Tract (CST) from a seed
region, which is indicated with a dashed black line in Fig. \ref{fig:CST}. This tract is
known for its huge lateral spread, which is normally difficult to recover. The
novel proposed consensus approach increases the completeness of the lateral
spread especially for the selection model dramatically. While the selection
model is only able to recover a few single streamlines within this area the
consensus approach improves this area significantly. The other approaches
were able to reconstruct the lateral spread also without the consensus approach
quiet well. However, the spread is denser within the consensus model. 
\begin{figure*}[t]
	\centering
	\includegraphics[width=\linewidth]{dir}
	\caption{Reconstructed fiber orientations of the different models, the
	red box in the left image denotes the position within the brain. Top row
shows models without consensus bootstrapping, bottom row with consensus
bootstrapping. Left: Averaging model. Right: selection model.}
	\label{fig:directions}
\end{figure*}

This observation can be further justified by inspecting the multi vector fields
on Fig. \ref{fig:directions}. There are many differences between the average and
selection model. The most obvious is, that the average model has three fibers in
each voxel, while the selection model contains often just one fiber in a voxel.
For the tracking approach this leads to much more spread as seen in the
reconstruction of the CST. However, this also leads to much more 'fuzziness'
within the average model reconstruction. Applying the consensus to the average
and selection model leads to quiet similar multi vector fields as it can be seen
in the red circle. Within this voxel the average and selection model are
dissimilar but the consensus model agrees. 

\subsection{Quantitative Comparison}
\begin{figure}[h]
	\centering
	\includegraphics[width=\linewidth]{Dice}
	\caption{Dice score averaged over all 12 patients tractwise for each
	model.}
	\label{fig:Dice}
\end{figure}
For a representative experiment we have reconstructed the corpus callosum (CC), the
cingulum (CG), the coricospinal tract (CST), the inferior fronto-occipital (IFO)
and the inferior longitudinal fasciculus (ILF), the optic radiation (OR), and
the superior longitudinal fasciculus (SLF). In the reference tragtographies, the
CC and SLF tracts were divided into subtracts, which we joined into a single
tract (CC) or one per hemisphere (SLF). 

The Dice scores of all tracts averaged over all 12 subjects are shown in Fig. 7.
In all cases the novel consensus approach increased the Dice score. Especially,
the selection model benefits greatly while the Dice score in the other models
increases only moderately. Overall the consensus selection model and the
consensus average model shares the highest Dice score. The 
\subsection{Computational Effort}
All experiments were computed on an Intel i9 with 3.3 GHz and 64 GB RAM. The
following durations are denoted in h:min:s or min:s.

The computation of a single bootstrap data sample took 1:10 on a single core,
multi-threaded fourth order fODF estimation took 4:33, the computation of the
selection model took 1:10, the computation of the averaging model took 1:30 and
the computation of the rank-3 model 0:40. The calculation of 100 bootstraps took
therefore 8:08:00, using all threads to create the bootstrap data as well as the
models. This also includes the computation of the consensus model, which on its
own took 11:00. The tracking itself is independent of the preprocessing steps.
It took approximately 2:30 for a bundle such it is shown in Figure xy on a
single thread including postprocessing.  


