\begin{abstract}
  
  Diffusion Magnetic Resonance Imaging (dMRI) tractography has the
  unique ability to reconstruct major white matter tracts
  non-invasively and is therefore widely used in neurosurgical
  planning and neuroscience. In this work, we reduce two sources of
  uncertainty within the tractography pipeline. The first one is the
  model uncertainty that arises in crossing fiber tractography, from
  having to estimate the number of relevant fiber compartments in each
  voxel. The second one is the data uncertainty that arises from
  measurement noise. We propose a mathematical framework to estimate
  model uncertainty, and we reduce this type of uncertainty with a
  model averaging approach that combines the fiber direction estimates
  from all candidate models, weighted by the posterior probability of
  the respective model. We use bootstrapping to estimate data
  uncertainty, and consolidate the fiber direction estimates from all
  bootstraps into a consensus model. We observe that a traditional
  model selection strategy frequently selects different models in
  different bootstraps. In this sense, the bootstrap consensus also
  reduces model uncertainty. Either approach increases the accuracy of
  crossing fiber tractography in multiple subjects, and combining them
  provides a small additional benefit.
  
  \textbf{Keywords:} diffusion MRI, tractography, uncertainty, model averaging, bootstrapping
%-------------------------------------------------------------------------
%  ACM CCS 1998
%  (see https://www.acm.org/publications/computing-classification-system/1998)
% \begin{classification} % according to https://www.acm.org/publications/computing-classification-system/1998
% \CCScat{Computer Graphics}{I.3.3}{Picture/Image Generation}{Line and curve generation}
% \end{classification}
%-------------------------------------------------------------------------
%  ACM CCS 2012
%   (see https://www.acm.org/publications/class-2012)
%The tool at \url{http://dl.acm.org/ccs.cfm} can be used to generate
% CCS codes.
%Example:
\begin{CCSXML}
<ccs2012>
   <concept>
       <concept_id>10010405.10010444</concept_id>
       <concept_desc>Applied computing~Life and medical sciences</concept_desc>
       <concept_significance>500</concept_significance>
       </concept>
   <concept>
       <concept_id>10002950.10003648.10003671</concept_id>
       <concept_desc>Mathematics of computing~Probabilistic algorithms</concept_desc>
       <concept_significance>400</concept_significance>
       </concept>
   <concept>
       <concept_id>10003120.10003145.10003146</concept_id>
       <concept_desc>Human-centered computing~Visualization techniques</concept_desc>
       <concept_significance>400</concept_significance>
       </concept>
 </ccs2012>
\end{CCSXML}

\ccsdesc[500]{Applied computing~Life and medical sciences}
\ccsdesc[400]{Mathematics of computing~Probabilistic algorithms}
\ccsdesc[400]{Human-centered computing~Visualization techniques}

\printccsdesc   
\end{abstract}  
