\begin{abstract}
	Diffusion Magnetic Resonance Imaging (dMRI) tractography has the unique
	ability to reconstruct major white matter tracts non-invasive and it
	therefore widely used in neurosurgical planning and in neuroscience.
	However, it is affected by various sources of uncertainties. Within our
	last work we evaluated the model uncertainty that arises by selecting a
	mathematical model to estimate multiple fiber orientations in a given
	voxel. We used this evaluation to propose a novel average model, where
	we have fused the information of all different models into one new
	model. This work extends our previous work, by evaluating the impact of
	measurement noise to the mathematical framework. We then use the newly
	derived uncertainty to reduce it by fuse all information into a new
	consensus model. Experiments on different white matter tracts in
	multiple subjects indicate that reducing the measurement uncertainty in
	this way increases the accuracy of crossing fiber tractography.
%-------------------------------------------------------------------------
%  ACM CCS 1998
%  (see https://www.acm.org/publications/computing-classification-system/1998)
% \begin{classification} % according to https://www.acm.org/publications/computing-classification-system/1998
% \CCScat{Computer Graphics}{I.3.3}{Picture/Image Generation}{Line and curve generation}
% \end{classification}
%-------------------------------------------------------------------------
%  ACM CCS 2012
%   (see https://www.acm.org/publications/class-2012)
%The tool at \url{http://dl.acm.org/ccs.cfm} can be used to generate
% CCS codes.
%Example:
\begin{CCSXML}
<ccs2012>
   <concept>
       <concept_id>10010405.10010444</concept_id>
       <concept_desc>Applied computing~Life and medical sciences</concept_desc>
       <concept_significance>500</concept_significance>
       </concept>
   <concept>
       <concept_id>10002950.10003648.10003671</concept_id>
       <concept_desc>Mathematics of computing~Probabilistic algorithms</concept_desc>
       <concept_significance>400</concept_significance>
       </concept>
   <concept>
       <concept_id>10003120.10003145.10003146</concept_id>
       <concept_desc>Human-centered computing~Visualization techniques</concept_desc>
       <concept_significance>400</concept_significance>
       </concept>
 </ccs2012>
\end{CCSXML}

\ccsdesc[500]{Applied computing~Life and medical sciences}
\ccsdesc[400]{Mathematics of computing~Probabilistic algorithms}
\ccsdesc[400]{Human-centered computing~Visualization techniques}

\printccsdesc   
\end{abstract}  
