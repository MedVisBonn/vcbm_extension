%        File: coverletter.tex
%     Created: Mi Mär 02 01:00  2022 C
% Last Change: Mi Mär 02 01:00  2022 C
%
\documentclass[a4paper]{article}
\begin{document}
\noindent
\textbf{Cover letter for the submission ``Model Averaging and Bootstrap Consensus Based Uncertainty Reduction in Diffusion MRI Tractography''}
\vskip\baselineskip
Dear editor, dear reviewers,

thank you again for the kind invitation to submit an extended version of our VCBM workshop paper ``Reducing Model Uncertainty in Crossing Fiber Tractography'' to Computer Graphics Forum. We are also thankful for your patience and for giving us the extra time that was needed to refine our original approach, to implement an additional, entirely new method, and to run several new and time-consuming experiments. We are confident that, compared to our previous publication, our current submission provides more than the requested 30\% additional benefit to the readers of your journal.

Our work improves the accuracy of diffusion MRI tractography by reducing different types of uncertainty. In particular, within our VCBM paper, we introduced model averaging as a means to reduce
the model uncertainty which arises from having to select a suitable fiber number at each point of the tracking process. Therefore, we used Bayesian model comparison to derive a probability for
each model, and fuse the information from all options into an average model. 

For our extension, we refined our implementation and experimental setup so that it yields even more accurate results, and provide a more detailed quantitative evaluation. Even more importantly, we introduce an additional approach, which we refer to as bootstrap consensus. It follows a similar information fusion idea as model averaging, but reduces data uncertainty. We demonstrate that it can be combined with model averaging, and that it further improves results. We also use it to study model uncertainty in more detail, in particular, how it interacts with the uncertainty from measurement noise, and how model averaging affects the variance in fiber estimates.

After finishing all these points, we are now looking forward to the constructive feedback from our peers.

Best regards

Johannes Grün, Gemma van der Voort, and Thomas Schultz

\end{document}


